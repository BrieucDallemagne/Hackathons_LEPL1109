\section*{Description of the project}
The air in cities is polluted by human activities. Vehicular emissions, heating systems, industries... You probably remember those times when the air is so polluted that it is recommended to stay at home, and not do too much sport outside. It is often related to specific weather conditions (e.g. not enough wind to blow air pollution away).

\bigskip

Besides, one could wonder: is it healthy to go running in a city? Do the benefits of sport balance the fact that you are breathing polluted air? Studies show that the answer is yes, as long as you avoid busy roads with dense traffic.

\bigskip

Air quality is a crucial issue in our modern society, and its effect is poorly anchored in common knowledge. As engineers of tomorrow, you must be aware of the impact of thermic engines and industries on human health.

\bigskip

In this hackathon, we will learn to quantify air quality. Wikipedia says that smog (or smoke fog) is composed of nitrogen oxides, sulfur oxide, ozone, smoke and other particulates. How to quantify from those different features?

\bigskip

People use the Air Quality Index (AQI). This is a natural number running from 0 to 500+, the lower the better. AQI accounts for the concentration of important pollutants and particles. We usually distinguish 6 classes of air quality, from good to severe (see table below). In this hackathon, we will classify the AQI between only 2 classes: good or bad.

\begin{figure}[h]
	\centering
	\includegraphics[width=\textwidth]{sections/imgs/aqi.png}
\end{figure}

\section*{Objective}
The project aims to train a binary classifier to estimate the air quality index (AQI) based on the air concentration of certain pollutants. Note that the AQI is initially defined as a natural number between 0 and 500+. However, to better address the problem, we propose to classify the air quality as poor (labeled 0) or good (labeled 1).   


